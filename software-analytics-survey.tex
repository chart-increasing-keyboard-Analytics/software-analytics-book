\documentclass[]{book}
\usepackage{lmodern}
\usepackage{amssymb,amsmath}
\usepackage{ifxetex,ifluatex}
\usepackage{fixltx2e} % provides \textsubscript
\ifnum 0\ifxetex 1\fi\ifluatex 1\fi=0 % if pdftex
  \usepackage[T1]{fontenc}
  \usepackage[utf8]{inputenc}
\else % if luatex or xelatex
  \ifxetex
    \usepackage{mathspec}
  \else
    \usepackage{fontspec}
  \fi
  \defaultfontfeatures{Ligatures=TeX,Scale=MatchLowercase}
\fi
% use upquote if available, for straight quotes in verbatim environments
\IfFileExists{upquote.sty}{\usepackage{upquote}}{}
% use microtype if available
\IfFileExists{microtype.sty}{%
\usepackage{microtype}
\UseMicrotypeSet[protrusion]{basicmath} % disable protrusion for tt fonts
}{}
\usepackage[margin=1in]{geometry}
\usepackage{hyperref}
\hypersetup{unicode=true,
            pdftitle={A Literature Survey of Software Analytics},
            pdfauthor={Moritz Beller, IN4334 2018 TU Delft},
            pdfborder={0 0 0},
            breaklinks=true}
\urlstyle{same}  % don't use monospace font for urls
\usepackage{natbib}
\bibliographystyle{apalike}
\usepackage{longtable,booktabs}
\usepackage{graphicx,grffile}
\makeatletter
\def\maxwidth{\ifdim\Gin@nat@width>\linewidth\linewidth\else\Gin@nat@width\fi}
\def\maxheight{\ifdim\Gin@nat@height>\textheight\textheight\else\Gin@nat@height\fi}
\makeatother
% Scale images if necessary, so that they will not overflow the page
% margins by default, and it is still possible to overwrite the defaults
% using explicit options in \includegraphics[width, height, ...]{}
\setkeys{Gin}{width=\maxwidth,height=\maxheight,keepaspectratio}
\IfFileExists{parskip.sty}{%
\usepackage{parskip}
}{% else
\setlength{\parindent}{0pt}
\setlength{\parskip}{6pt plus 2pt minus 1pt}
}
\setlength{\emergencystretch}{3em}  % prevent overfull lines
\providecommand{\tightlist}{%
  \setlength{\itemsep}{0pt}\setlength{\parskip}{0pt}}
\setcounter{secnumdepth}{5}
% Redefines (sub)paragraphs to behave more like sections
\ifx\paragraph\undefined\else
\let\oldparagraph\paragraph
\renewcommand{\paragraph}[1]{\oldparagraph{#1}\mbox{}}
\fi
\ifx\subparagraph\undefined\else
\let\oldsubparagraph\subparagraph
\renewcommand{\subparagraph}[1]{\oldsubparagraph{#1}\mbox{}}
\fi

%%% Use protect on footnotes to avoid problems with footnotes in titles
\let\rmarkdownfootnote\footnote%
\def\footnote{\protect\rmarkdownfootnote}

%%% Change title format to be more compact
\usepackage{titling}

% Create subtitle command for use in maketitle
\newcommand{\subtitle}[1]{
  \posttitle{
    \begin{center}\large#1\end{center}
    }
}

\setlength{\droptitle}{-2em}

  \title{A Literature Survey of Software Analytics}
    \pretitle{\vspace{\droptitle}\centering\huge}
  \posttitle{\par}
    \author{Moritz Beller, IN4334 2018 TU Delft}
    \preauthor{\centering\large\emph}
  \postauthor{\par}
      \predate{\centering\large\emph}
  \postdate{\par}
    \date{2018-09-14}

\usepackage{booktabs}
\usepackage{amsthm}
\makeatletter
\def\thm@space@setup{%
  \thm@preskip=8pt plus 2pt minus 4pt
  \thm@postskip=\thm@preskip
}
\makeatother

\begin{document}
\maketitle

{
\setcounter{tocdepth}{1}
\tableofcontents
}
\chapter{Preamble}\label{intro}

The book you see in front of you is the outcome of an eight week seminar
run by the Software Engineering Research Group (SERG) at TU Delft. We
have split up the novel area of Software Analytics into several sub
topics. Every chapter addresses one such sub-topic of Software Analytics
and is the outcome of a systematic literature review a laborious team of
3-4 students performed.

With this book, we hope to structure the new field of Software Analytics
and show how it is related to many long existing research fields.

\emph{Moritz Beller}

\section{License}\label{license}

\includegraphics{figures/cc-nc-sa.png} This book is copyrighted 2018 by
TU Delft and its respective authors and distributed under a
\href{https://creativecommons.org/licenses/by-nc-sa/4.0/}{CC BY-NC-SA
4.0 license}

\chapter{A contemporary view on Software
Analytics}\label{a-contemporary-view-on-software-analytics}

\section{What is Software Analytics?}\label{what-is-software-analytics}

\section{A list of Software Analytics
Sub-Topics}\label{a-list-of-software-analytics-sub-topics}

\chapter{Sample Sub-Topic}\label{sample-sub-topic}

This is an example for the deliverable every group works on. Every group
works on one independent chapter (starting as one Rmd file).

\chapter{Final Words}\label{final-words}

We have finished a nice book on Software Analytics.

\chapter{Build analytics}\label{build-analytics}

\section{Background}\label{background}

When building a project from source code to executables everything
should go smoothly. This is not always the case, a build can break for
several reasons. This chapter will give an overview of research done on
build scripts and continuous integration.

\section{Research Questions}\label{research-questions}

\begin{itemize}
\tightlist
\item
  What are causes of a broken build?
\item
  With which goals is continuous integration applied?
\item
  What can be used to effectively fix a broken build?
\end{itemize}

\section{Search Strategy}\label{search-strategy}

Using the initial seed consisting of \citet{bird2017predicting},
\citet{beller2017oops}, \citet{rausch2017empirical},
\citet{beller2017travistorrent}, \citet{pinto2018work},
\citet{zhao2017impact}, \citet{widder2018m} and \citet{hilton2016usage}
we used references and similar keywords to find new papers to analyze.

\section{Study Selection}\label{study-selection}

Through this we found the following papers

\section{Summary of papers}\label{summary-of-papers}

\subsection{\texorpdfstring{\citet{bird2017predicting}}{@bird2017predicting}}\label{bird2017predicting}

This is a US patent grant for a method of predicting software build
errors. This patent is owned my Microsoft. Using logistic regression a
prediction can be made on the probability of a build failing. Using this
method build errors can be better anticipated, which decreases the time
until the build works again.

\subsection{\texorpdfstring{\citet{vassallo2017tale}}{@vassallo2017tale}}\label{vassallo2017tale}

Discusses the difference in failures on continuous integration between
open source software (OSS) and industrial software projects. For this
349 Java OSS projects and 418 project from ING Nederland, a financial
organisation.

Using cluser analysis it was observed that both kinds of projects share
similar build failures, but in other cases very different patterns
emerge.

\subsection{\texorpdfstring{\citet{hassan2018hirebuild}}{@hassan2018hirebuild}}\label{hassan2018hirebuild}

This paper uses TravisTorrent (\citet{beller2017travistorrent}) to show
that 22\% of code commits include changes in build script files to keep
the build working or to fix the build.

In the paper a tool is proposed to automatically fix build failures
based on previous changes.

\subsection{\texorpdfstring{\citet{vassallo2018break}}{@vassallo2018break}}\label{vassallo2018break}

This paper proposes a tool called \textsc{BART} to help developers fix
build errors. This tool eliminates the need to browse error logs which
can be very long by generating a summary of the failure with useful
information.

\subsection{\texorpdfstring{\citet{zampetti2017open}}{@zampetti2017open}}\label{zampetti2017open}

This paper studies the usage of static analysis tools in 20 Java open
source software projects hosted on GitHub and using Travic CI as
continuous integration infrastructure. There is investigated which tools
are being used, what types of issues make the build fail or raise
warnings and how is responded to broken builds.

\bibliography{book.bib}


\end{document}
