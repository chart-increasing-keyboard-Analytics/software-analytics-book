\documentclass[]{book}
\usepackage{lmodern}
\usepackage{amssymb,amsmath}
\usepackage{ifxetex,ifluatex}
\usepackage{fixltx2e} % provides \textsubscript
\ifnum 0\ifxetex 1\fi\ifluatex 1\fi=0 % if pdftex
  \usepackage[T1]{fontenc}
  \usepackage[utf8]{inputenc}
\else % if luatex or xelatex
  \ifxetex
    \usepackage{mathspec}
  \else
    \usepackage{fontspec}
  \fi
  \defaultfontfeatures{Ligatures=TeX,Scale=MatchLowercase}
\fi
% use upquote if available, for straight quotes in verbatim environments
\IfFileExists{upquote.sty}{\usepackage{upquote}}{}
% use microtype if available
\IfFileExists{microtype.sty}{%
\usepackage{microtype}
\UseMicrotypeSet[protrusion]{basicmath} % disable protrusion for tt fonts
}{}
\usepackage[margin=1in]{geometry}
\usepackage{hyperref}
\hypersetup{unicode=true,
            pdftitle={A Literature Survey of Software Analytics},
            pdfauthor={Moritz Beller, IN4334 2018 TU Delft},
            pdfborder={0 0 0},
            breaklinks=true}
\urlstyle{same}  % don't use monospace font for urls
\usepackage{natbib}
\bibliographystyle{apalike}
\usepackage{longtable,booktabs}
\usepackage{graphicx,grffile}
\makeatletter
\def\maxwidth{\ifdim\Gin@nat@width>\linewidth\linewidth\else\Gin@nat@width\fi}
\def\maxheight{\ifdim\Gin@nat@height>\textheight\textheight\else\Gin@nat@height\fi}
\makeatother
% Scale images if necessary, so that they will not overflow the page
% margins by default, and it is still possible to overwrite the defaults
% using explicit options in \includegraphics[width, height, ...]{}
\setkeys{Gin}{width=\maxwidth,height=\maxheight,keepaspectratio}
\IfFileExists{parskip.sty}{%
\usepackage{parskip}
}{% else
\setlength{\parindent}{0pt}
\setlength{\parskip}{6pt plus 2pt minus 1pt}
}
\setlength{\emergencystretch}{3em}  % prevent overfull lines
\providecommand{\tightlist}{%
  \setlength{\itemsep}{0pt}\setlength{\parskip}{0pt}}
\setcounter{secnumdepth}{5}
% Redefines (sub)paragraphs to behave more like sections
\ifx\paragraph\undefined\else
\let\oldparagraph\paragraph
\renewcommand{\paragraph}[1]{\oldparagraph{#1}\mbox{}}
\fi
\ifx\subparagraph\undefined\else
\let\oldsubparagraph\subparagraph
\renewcommand{\subparagraph}[1]{\oldsubparagraph{#1}\mbox{}}
\fi

%%% Use protect on footnotes to avoid problems with footnotes in titles
\let\rmarkdownfootnote\footnote%
\def\footnote{\protect\rmarkdownfootnote}

%%% Change title format to be more compact
\usepackage{titling}

% Create subtitle command for use in maketitle
\newcommand{\subtitle}[1]{
  \posttitle{
    \begin{center}\large#1\end{center}
    }
}

\setlength{\droptitle}{-2em}

  \title{A Literature Survey of Software Analytics}
    \pretitle{\vspace{\droptitle}\centering\huge}
  \posttitle{\par}
    \author{Moritz Beller, IN4334 2018 TU Delft}
    \preauthor{\centering\large\emph}
  \postauthor{\par}
      \predate{\centering\large\emph}
  \postdate{\par}
    \date{2018-09-20}

\usepackage{booktabs}
\usepackage{amsthm}
\makeatletter
\def\thm@space@setup{%
  \thm@preskip=8pt plus 2pt minus 4pt
  \thm@postskip=\thm@preskip
}
\makeatother

\begin{document}
\maketitle

{
\setcounter{tocdepth}{1}
\tableofcontents
}
\chapter{Preamble}\label{intro}

The book you see in front of you is the outcome of an eight week seminar
run by the Software Engineering Research Group (SERG) at TU Delft. We
have split up the novel area of Software Analytics into several sub
topics. Every chapter addresses one such sub-topic of Software Analytics
and is the outcome of a systematic literature review a laborious team of
3-4 students performed.

With this book, we hope to structure the new field of Software Analytics
and show how it is related to many long existing research fields.

\emph{Moritz Beller}

\section{License}\label{license}

\includegraphics{figures/cc-nc-sa.png} This book is copyrighted 2018 by
TU Delft and its respective authors and distributed under a
\href{https://creativecommons.org/licenses/by-nc-sa/4.0/}{CC BY-NC-SA
4.0 license}

\chapter{A contemporary view on Software
Analytics}\label{a-contemporary-view-on-software-analytics}

\section{What is Software Analytics?}\label{what-is-software-analytics}

\section{A list of Software Analytics
Sub-Topics}\label{a-list-of-software-analytics-sub-topics}

\chapter{Build analytics}\label{build-analytics}

\section{Background}\label{background}

When building a project from source code to executables everything
should go smoothly. This is not always the case, a build can break for
several reasons. This chapter will give an overview of research done on
build configurations and continuous integration.

\section{Research Questions}\label{research-questions}

\textbf{RQ1} What is the current state of the art in the field of build
analytics? \textbf{RQ2} What is the current state of practice in the
field of build analytics? \textbf{RQ3} What future research can we
expect in the field of build analytics?

\section{Search Strategy}\label{search-strategy}

Using the initial seed consisting of \citet{bird2017predicting},
\citet{beller2017oops}, \citet{rausch2017empirical},
\citet{beller2017travistorrent}, \citet{pinto2018work},
\citet{zhao2017impact}, \citet{widder2018m} and \citet{hilton2016usage}
we used references to find new papers to analyze.

\section{Study Selection}\label{study-selection}

Through this we found the following papers

\section{Summary of papers}\label{summary-of-papers}

\subsection{\texorpdfstring{\citet{bird2017predicting}}{@bird2017predicting}}\label{bird2017predicting}

\emph{Initial Seed}

This is a US patent grant for a method of predicting software build
errors. This patent is owned by Microsoft. Using logistic regression a
prediction can be made on the probability of a build failing. Using this
method build errors can be better anticipated, which decreases the time
until the build works again.

\subsection{\texorpdfstring{\citet{beller2017oops}}{@beller2017oops}}\label{beller2017oops}

\emph{Initial Seed}

This paper explores data from Travis CI\footnote{See
  \url{https://travis-ci.org}} on a large scale by analyzing 2,640,825
build logs of Java and Ruby builds. It uses \textsc{TravisTorrent} as a
data source. It is found that the number one reason for failing builds
it test failure. It also explores differences in testing between Java
and Ruby.

\subsection{\texorpdfstring{\citet{rausch2017empirical}}{@rausch2017empirical}}\label{rausch2017empirical}

\emph{Initial Seed}

A stuy on the build results of 14 open source software Java projects. It
is similar to \citet{beller2017oops}, albeit on a smaller scale. It does
go more in depth on the result and changes over time.

\subsection{\texorpdfstring{\citet{beller2017travistorrent}}{@beller2017travistorrent}}\label{beller2017travistorrent}

\emph{Initial Seed}

This paper introduces \textsc{TravisTorrent}, a dataset containing
analyzed builds from more than 1,000 projects. This data is freely
downloadable from the internet. It uses \textsc{GHTorrent} to link the
information from travis to commits on GitHub.

\subsection{\texorpdfstring{\citet{pinto2018work}}{@pinto2018work}}\label{pinto2018work}

\emph{Initial Seed}

This paper is a survey amongst Travis CI users. It found that users are
not sure whether a job failure represents a failure or not, that
inadequate testing is the most common (technical) reason for build
breakage and that people feel that there is a false sense of confidence
when blindly trusing tests.

\subsection{\texorpdfstring{\citet{zhao2017impact}}{@zhao2017impact}}\label{zhao2017impact}

\emph{Initial Seed}

This paper analyzed approximately 160,000 projects written in seven
different programming languages. It notes that adoption of CI is often
part of a reorganization. It collected information on the differences
before and after adoption of CI. There is also a survey amongst
developers to learn about their experiences in adopting Travis CI.

\subsection{\texorpdfstring{\citet{widder2018m}}{@widder2018m}}\label{widder2018m}

\emph{Initial Seed}

This paper analyzes what factors have impact on abandonment of Travis.
They find that increased build complexity reduces the chance of
abandonment, but larger projects abandon at a higher rate and that a
project's language has significant but varying effect. A surprising
result is that metrics of configuration attempts and knowledge
dispersion in the project don't affect the rate of abandonment.

\subsection{\texorpdfstring{\citet{hilton2016usage}}{@hilton2016usage}}\label{hilton2016usage}

\emph{Initial Seed}

This paper explores which CI system developers use, how developers use
CI and why developers use CI. For this it analyzes data from Github,
Travis CI and it conducts a developer survey. It finds that projects
using CI release twice as often, accept pull requests faster and have
developers who are less worried about breaking the build.

\subsection{\texorpdfstring{\citet{vassallo2017tale}}{@vassallo2017tale}}\label{vassallo2017tale}

\emph{References \citet{beller2017oops} }

This paper discusses the difference in failures on continuous
integration between open source software (OSS) and industrial software
projects. For this 349 Java OSS projects and 418 project from ING
Nederland, a financial organization.

Using cluser analysis it was observed that both kinds of projects share
similar build failures, but in other cases very different patterns
emerge.

\subsection{\texorpdfstring{\citet{hassan2018hirebuild}}{@hassan2018hirebuild}}\label{hassan2018hirebuild}

\emph{References \citet{beller2017travistorrent} }

This paper uses TravisTorrent (\citet{beller2017travistorrent}) to show
that 22\% of code commits include changes in build script files to keep
the build working or to fix the build.

In the paper a tool is proposed to automatically fix build failures
based on previous changes.

\subsection{\texorpdfstring{\citet{vassallo2018break}}{@vassallo2018break}}\label{vassallo2018break}

\emph{References \citet{beller2017oops}, \citet{rausch2017empirical} }

This paper proposes a tool called \textsc{BART} to help developers fix
build errors. This tool eliminates the need to browse error logs which
can be very long by generating a summary of the failure with useful
information.

\subsection{\texorpdfstring{\citet{zampetti2017open}}{@zampetti2017open}}\label{zampetti2017open}

\emph{Referenced by \citet{vassallo2018break} }

This paper studies the usage of static analysis tools in 20 Java open
source software projects hosted on GitHub and using Travic CI as
continuous integration infrastructure. There is investigated which tools
are being used, what types of issues make the build fail or raise
warnings and how is responded to broken builds.

\section{What is the current state of practice in the field of build
analytics?}\label{what-is-the-current-state-of-practice-in-the-field-of-build-analytics}

In this section, I will examine scientific papers to analyse the current
trend of build analytics in the software development industry.

\subsection{Continuous Integration by Martin
Fowler}\label{continuous-integration-by-martin-fowler}

In this paper, Martin talks about the current state of the software
industry in terms of Continuous Integration (CI) and comments on the
practises required to implement CI effectively. He talks about his
experience working for a large English electronics company where the
development of a project took two years and the integration process took
several months. Integration is a long and unpredictable process. Martin
suggested this approach and that the two most common reactions he got
were: ``it can't work (here)'' or ``doing it won't make much
difference''. He expresses that most engineers don't know how simple the
process can be of setting the CI framework up. In this way, we get a
glimpse into the practises popular within the industry regarding build
analytics.

\subsection{Usage, Costs, and Benefits of Continuous Integration in
Open-Source
Projects}\label{usage-costs-and-benefits-of-continuous-integration-in-open-source-projects}

\chapter{Sample Sub-Topic}\label{sample-sub-topic}

\emph{This is an example for the deliverable every group works on. Every
group works on one independent chapter (starting as one Rmd file).}

\section{Motivation}\label{motivation}

\emph{A short introduction about why the topic you are working on is
interesting.}

The RQs that everyone should be aiming at are:

\begin{itemize}
\tightlist
\item
  \textbf{RQ1} Current state of the art in software analytics for
  \emph{your topic }:

  \begin{itemize}
  \tightlist
  \item
    Topics that are being explored
  \item
    Research methods, tools and datasets being used
  \item
    Main research findings, aggregated
  \end{itemize}
\item
  \textbf{RQ2} Current state of practice in software analytics for
  \emph{your topic }:

  \begin{itemize}
  \tightlist
  \item
    Tools and companies creating / employing them
  \item
    Case studies and their findings
  \end{itemize}
\item
  \textbf{RQ3} Open challenges and future research required
\end{itemize}

\section{Research protocol}\label{research-protocol}

\emph{Here, you describe the details of applying Kitchenham's survey
method for your topic, including search queries, fact extraction, coding
process and an intial groupping of the papers that you will be
analyzing.}

\section{Answers}\label{answers}

\emph{Aggregated answers to the RQs, per RQ. You need:}

\begin{itemize}
\tightlist
\item
  For \textbf{RQ1}

  \begin{itemize}
  \tightlist
  \item
    Topics that are being explored
  \item
    Research methods, tools and datasets being used
  \item
    Main research findings, aggregated
  \end{itemize}
\item
  For \textbf{RQ2} :

  \begin{itemize}
  \tightlist
  \item
    Tools and companies creating / employing them
  \item
    Case studies and their findings
  \end{itemize}
\item
  For \textbf{RQ3}:

  \begin{itemize}
  \tightlist
  \item
    List of challenges
  \item
    An aggregated set of open research items, as described in the papers
  \item
    Research questions that emerge from the synthesis of the presented
    works
  \end{itemize}
\end{itemize}

\chapter{Final Words}\label{final-words}

We have finished a nice book on Software Analytics.

\bibliography{book.bib}


\end{document}
