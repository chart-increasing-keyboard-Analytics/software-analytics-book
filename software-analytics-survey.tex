\documentclass[]{book}
\usepackage{lmodern}
\usepackage{amssymb,amsmath}
\usepackage{ifxetex,ifluatex}
\usepackage{fixltx2e} % provides \textsubscript
\ifnum 0\ifxetex 1\fi\ifluatex 1\fi=0 % if pdftex
  \usepackage[T1]{fontenc}
  \usepackage[utf8]{inputenc}
\else % if luatex or xelatex
  \ifxetex
    \usepackage{mathspec}
  \else
    \usepackage{fontspec}
  \fi
  \defaultfontfeatures{Ligatures=TeX,Scale=MatchLowercase}
\fi
% use upquote if available, for straight quotes in verbatim environments
\IfFileExists{upquote.sty}{\usepackage{upquote}}{}
% use microtype if available
\IfFileExists{microtype.sty}{%
\usepackage{microtype}
\UseMicrotypeSet[protrusion]{basicmath} % disable protrusion for tt fonts
}{}
\usepackage[margin=1in]{geometry}
\usepackage{hyperref}
\hypersetup{unicode=true,
            pdftitle={A Literature Survey of Software Analytics},
            pdfauthor={Moritz Beller, IN4334 2018 TU Delft},
            pdfborder={0 0 0},
            breaklinks=true}
\urlstyle{same}  % don't use monospace font for urls
\usepackage{natbib}
\bibliographystyle{apalike}
\usepackage{longtable,booktabs}
\usepackage{graphicx,grffile}
\makeatletter
\def\maxwidth{\ifdim\Gin@nat@width>\linewidth\linewidth\else\Gin@nat@width\fi}
\def\maxheight{\ifdim\Gin@nat@height>\textheight\textheight\else\Gin@nat@height\fi}
\makeatother
% Scale images if necessary, so that they will not overflow the page
% margins by default, and it is still possible to overwrite the defaults
% using explicit options in \includegraphics[width, height, ...]{}
\setkeys{Gin}{width=\maxwidth,height=\maxheight,keepaspectratio}
\IfFileExists{parskip.sty}{%
\usepackage{parskip}
}{% else
\setlength{\parindent}{0pt}
\setlength{\parskip}{6pt plus 2pt minus 1pt}
}
\setlength{\emergencystretch}{3em}  % prevent overfull lines
\providecommand{\tightlist}{%
  \setlength{\itemsep}{0pt}\setlength{\parskip}{0pt}}
\setcounter{secnumdepth}{5}
% Redefines (sub)paragraphs to behave more like sections
\ifx\paragraph\undefined\else
\let\oldparagraph\paragraph
\renewcommand{\paragraph}[1]{\oldparagraph{#1}\mbox{}}
\fi
\ifx\subparagraph\undefined\else
\let\oldsubparagraph\subparagraph
\renewcommand{\subparagraph}[1]{\oldsubparagraph{#1}\mbox{}}
\fi

%%% Use protect on footnotes to avoid problems with footnotes in titles
\let\rmarkdownfootnote\footnote%
\def\footnote{\protect\rmarkdownfootnote}

%%% Change title format to be more compact
\usepackage{titling}

% Create subtitle command for use in maketitle
\newcommand{\subtitle}[1]{
  \posttitle{
    \begin{center}\large#1\end{center}
    }
}

\setlength{\droptitle}{-2em}

  \title{A Literature Survey of Software Analytics}
    \pretitle{\vspace{\droptitle}\centering\huge}
  \posttitle{\par}
    \author{Moritz Beller, IN4334 2018 TU Delft}
    \preauthor{\centering\large\emph}
  \postauthor{\par}
      \predate{\centering\large\emph}
  \postdate{\par}
    \date{2018-09-24}

\usepackage{booktabs}
\usepackage{amsthm}
\makeatletter
\def\thm@space@setup{%
  \thm@preskip=8pt plus 2pt minus 4pt
  \thm@postskip=\thm@preskip
}
\makeatother

\begin{document}
\maketitle

{
\setcounter{tocdepth}{1}
\tableofcontents
}
\chapter{Preamble}\label{intro}

The book you see in front of you is the outcome of an eight week seminar
run by the Software Engineering Research Group (SERG) at TU Delft. We
have split up the novel area of Software Analytics into several sub
topics. Every chapter addresses one such sub-topic of Software Analytics
and is the outcome of a systematic literature review a laborious team of
3-4 students performed.

With this book, we hope to structure the new field of Software Analytics
and show how it is related to many long existing research fields.

\emph{Moritz Beller}

\section{License}\label{license}

\includegraphics{figures/cc-nc-sa.png} This book is copyrighted 2018 by
TU Delft and its respective authors and distributed under a
\href{https://creativecommons.org/licenses/by-nc-sa/4.0/}{CC BY-NC-SA
4.0 license}

\chapter{A contemporary view on Software
Analytics}\label{a-contemporary-view-on-software-analytics}

\section{What is Software Analytics?}\label{what-is-software-analytics}

\section{A list of Software Analytics
Sub-Topics}\label{a-list-of-software-analytics-sub-topics}

\chapter{Build analytics}\label{build-analytics}

\section{Motivation}\label{motivation}

Ideally, when building a project from source code to executable, the
process should be fast and without any errors. Unfortunately, this is
not always the case and automated builds results notify developers of
compile errors, missing dependencies, broken functionality and many
other problems. This chapter is aimed to give an overview of the effort
made in build analytics field and Continuous Integration (CI) as an
increasingly common development practice in many projects.

\section{Research Questions}\label{research-questions}

\textbf{RQ1} What is the current state of the art in the field of build
analytics? \textbf{RQ2} What is the current state of practice in the
field of build analytics? \textbf{RQ3} What future research can we
expect in the field of build analytics?

\section{Search Strategy}\label{search-strategy}

Using the initial seed consisting of \citet{bird2017predicting},
\citet{beller2017oops}, \citet{rausch2017empirical},
\citet{beller2017travistorrent}, \citet{pinto2018work},
\citet{zhao2017impact}, \citet{widder2018m} and \citet{hilton2016usage}
we used references to find new papers to analyze.

\section{Study Selection}\label{study-selection}

Through this we found the following papers

\section{Summary of papers}\label{summary-of-papers}

\subsection{\texorpdfstring{\citet{bird2017predicting}}{@bird2017predicting}}\label{bird2017predicting}

\emph{Initial Seed}

This is a US patent grant for a method of predicting software build
errors. This patent is owned by Microsoft. Using logistic regression a
prediction can be made on the probability of a build failing. Using this
method build errors can be better anticipated, which decreases the time
until the build works again.

\subsection{\texorpdfstring{\citet{beller2017oops}}{@beller2017oops}}\label{beller2017oops}

\emph{Initial Seed}

This paper explores data from Travis CI\footnote{See
  \url{https://travis-ci.org}} on a large scale by analyzing 2,640,825
build logs of Java and Ruby builds. It uses \textsc{TravisTorrent} as a
data source. It is found that the number one reason for failing builds
it test failure. It also explores differences in testing between Java
and Ruby.

\subsection{\texorpdfstring{\citet{rausch2017empirical}}{@rausch2017empirical}}\label{rausch2017empirical}

\emph{Initial Seed}

A stuy on the build results of 14 open source software Java projects. It
is similar to \citet{beller2017oops}, albeit on a smaller scale. It does
go more in depth on the result and changes over time.

\subsection{\texorpdfstring{\citet{beller2017travistorrent}}{@beller2017travistorrent}}\label{beller2017travistorrent}

\emph{Initial Seed}

This paper introduces \textsc{TravisTorrent}, a dataset containing
analyzed builds from more than 1,000 projects. This data is freely
downloadable from the internet. It uses \textsc{GHTorrent} to link the
information from travis to commits on GitHub.

\subsection{\texorpdfstring{\citet{pinto2018work}}{@pinto2018work}}\label{pinto2018work}

\emph{Initial Seed}

This paper is a survey amongst Travis CI users. It found that users are
not sure whether a job failure represents a failure or not, that
inadequate testing is the most common (technical) reason for build
breakage and that people feel that there is a false sense of confidence
when blindly trusing tests.

\subsection{\texorpdfstring{\citet{zhao2017impact}}{@zhao2017impact}}\label{zhao2017impact}

\emph{Initial Seed}

This paper analyzed approximately 160,000 projects written in seven
different programming languages. It notes that adoption of CI is often
part of a reorganization. It collected information on the differences
before and after adoption of CI. There is also a survey amongst
developers to learn about their experiences in adopting Travis CI.

\subsection{\texorpdfstring{\citet{widder2018m}}{@widder2018m}}\label{widder2018m}

\emph{Initial Seed}

This paper analyzes what factors have impact on abandonment of Travis.
They find that increased build complexity reduces the chance of
abandonment, but larger projects abandon at a higher rate and that a
project's language has significant but varying effect. A surprising
result is that metrics of configuration attempts and knowledge
dispersion in the project do not affect the rate of abandonment.

\subsection{\texorpdfstring{\citet{hilton2016usage}}{@hilton2016usage}}\label{hilton2016usage}

\emph{Initial Seed}

This paper explores which CI system developers use, how developers use
CI and why developers use CI. For this it analyzes data from Github,
Travis CI and it conducts a developer survey. It finds that projects
using CI release twice as often, accept pull requests faster and have
developers who are less worried about breaking the build.

\subsection{\texorpdfstring{\citet{vassallo2017tale}}{@vassallo2017tale}}\label{vassallo2017tale}

\emph{References \citet{beller2017oops} }

This paper discusses the difference in failures on continuous
integration between open source software (OSS) and industrial software
projects. For this 349 Java OSS projects and 418 project from ING
Nederland, a financial organization.

Using cluser analysis it was observed that both kinds of projects share
similar build failures, but in other cases very different patterns
emerge.

\subsection{\texorpdfstring{\citet{hassan2018hirebuild}}{@hassan2018hirebuild}}\label{hassan2018hirebuild}

\emph{References \citet{beller2017travistorrent} }

This paper uses TravisTorrent (\citet{beller2017travistorrent}) to show
that 22\% of code commits include changes in build script files to keep
the build working or to fix the build.

In the paper a tool is proposed to automatically fix build failures
based on previous changes.

\subsection{\texorpdfstring{\citet{vassallo2018break}}{@vassallo2018break}}\label{vassallo2018break}

\emph{References \citet{beller2017oops}, \citet{rausch2017empirical} }

This paper proposes a tool called \textsc{BART} to help developers fix
build errors. This tool eliminates the need to browse error logs which
can be very long by generating a summary of the failure with useful
information.

\subsection{\texorpdfstring{\citet{zampetti2017open}}{@zampetti2017open}}\label{zampetti2017open}

\emph{Referenced by \citet{vassallo2018break} }

This paper studies the usage of static analysis tools in 20 Java open
source software projects hosted on GitHub and using Travic CI as
continuous integration infrastructure. There is investigated which tools
are being used, what types of issues make the build fail or raise
warnings and how is responded to broken builds.

\subsection{\texorpdfstring{\citet{baltes2018no}}{@baltes2018no}}\label{baltes2018no}

\emph{Google Scholar search term
\texttt{Github\ "Continuous\ Integration"}, papers from 2018}

This paper analyses 93 GitHub projects before and after adoption of
Travis CI. It finds only one non-negligible effect, an increasing merge
ratio, meaning that more merging commits in relation to all commits
after a project started using Travis CI. But the paper also shows that
this effect can be seen on projects not adopting CI. It shows the
importance of having a proper dataset with as little bias as possible.

\section{What is the current state of the art in the field of build
analytics?}\label{what-is-the-current-state-of-the-art-in-the-field-of-build-analytics}

\section{What is the current state of practice in the field of build
analytics?}\label{what-is-the-current-state-of-practice-in-the-field-of-build-analytics}

In this section, I will examine scientific papers to analyse the current
trend of build analytics in the software development industry.

\subsection{\texorpdfstring{\citet{fowler2006continuous}}{@fowler2006continuous}}\label{fowler2006continuous}

In this paper, Martin talks about the current state of the software
industry in terms of Continuous Integration (CI) and comments on the
practises required to implement CI effectively. He talks about his
experience working for a large English electronics company where the
development of a project took two years and the integration process took
several months. Integration is a long and unpredictable process. Martin
suggested this approach and that the two most common reactions he got
were: ``it can't work (here)'' or ``doing it won't make much
difference''. He expresses that most engineers don't know how simple the
process can be of setting the CI framework up. In this way, we get a
glimpse into the practises popular within the industry regarding build
analytics.

\subsection{\texorpdfstring{\citet{hilton2016usage}}{@hilton2016usage}}\label{hilton2016usage-1}

This paper examines the usage, costs and benefits of Continuous
Integration. A survey conducted in open-source projects indicated that
40\% of all projects used CI. Of the projects that used CI, 90\% used
Travis for their CI services. They also determine that the more popular
projects use CI but there is no correlation between the popularity of
language and usage of CI. It also observes that the median project
introduces CI a year into development. The paper claims that CI is
widely used in practise nowadays and CI adoption rates will increase
even further in the future.

\subsection{\texorpdfstring{\citet{rausch2017empirical}}{@rausch2017empirical}}\label{rausch2017empirical-1}

Version Control Systems (VCS) such as GitHub, and hosted build
automation platforms such as Travis, have made Continuous Integration is
widely available for projects of every size. This paper suggests that CI
is widely used and has improved the quality of processes and developed
software itself. However, the article suggests that there is little
known about the variety and frequency of errors that cause builds to
fail. It suggests that developers should eliminate flaky tests and
address common issues regularly such as broken interaction with
repositories to keep the build system healthy.

\subsection{\texorpdfstring{\citet{stolberg2009enabling}}{@stolberg2009enabling}}\label{stolberg2009enabling}

This paper defines CI as a key element in agile software development and
testing environment. It also uses Marin Fowler's practises of CI (as
discussed previously) and expresses the importance of CI in the software
industry.

\section{What future research can we expect in the field of build
analytics?}\label{what-future-research-can-we-expect-in-the-field-of-build-analytics}

Future research in build analytics branches in a couple of different
topics. \citet{pinto2018work} proposes to focus on getting a better
understanding of the users and why they might choose to abandon an
automatic build platform.

According to \citet{baltes2018no} future work could look into more
perspectives when analyzing commit data, for instance partitioning
commits by developer. It also notes the importance of more qualitative
research.

\chapter{Sample Sub-Topic}\label{sample-sub-topic}

\emph{This is an example for the deliverable every group works on. Every
group works on one independent chapter (starting as one Rmd file).}

\section{Motivation}\label{motivation-1}

\emph{A short introduction about why the topic you are working on is
interesting.}

The RQs that everyone should be aiming at are:

\begin{itemize}
\tightlist
\item
  \textbf{RQ1} Current state of the art in software analytics for
  \emph{your topic }:

  \begin{itemize}
  \tightlist
  \item
    Topics that are being explored
  \item
    Research methods, tools and datasets being used
  \item
    Main research findings, aggregated
  \end{itemize}
\item
  \textbf{RQ2} Current state of practice in software analytics for
  \emph{your topic }:

  \begin{itemize}
  \tightlist
  \item
    Tools and companies creating / employing them
  \item
    Case studies and their findings
  \end{itemize}
\item
  \textbf{RQ3} Open challenges and future research required
\end{itemize}

\section{Research protocol}\label{research-protocol}

\emph{Here, you describe the details of applying Kitchenham's survey
method for your topic, including search queries, fact extraction, coding
process and an intial groupping of the papers that you will be
analyzing.}

\section{Answers}\label{answers}

\emph{Aggregated answers to the RQs, per RQ. You need:}

\begin{itemize}
\tightlist
\item
  For \textbf{RQ1}

  \begin{itemize}
  \tightlist
  \item
    Topics that are being explored
  \item
    Research methods, tools and datasets being used
  \item
    Main research findings, aggregated
  \end{itemize}
\item
  For \textbf{RQ2} :

  \begin{itemize}
  \tightlist
  \item
    Tools and companies creating / employing them
  \item
    Case studies and their findings
  \end{itemize}
\item
  For \textbf{RQ3}:

  \begin{itemize}
  \tightlist
  \item
    List of challenges
  \item
    An aggregated set of open research items, as described in the papers
  \item
    Research questions that emerge from the synthesis of the presented
    works
  \end{itemize}
\end{itemize}

\chapter{Final Words}\label{final-words}

We have finished a nice book on Software Analytics.

\chapter{App Store analytics}\label{app-store-analytics}

\section{API change and fault proneness: A threat to the success of
Android
apps}\label{api-change-and-fault-proneness-a-threat-to-the-success-of-android-apps}

M. Linares-Vásquez, G. Bavota, C. Bernal-Cárdenas, M. Di Penta, R.
Oliveto, and D. Poshyvanyk, in Proceedings of the 2013 9th joint meeting
on foundations of software engineering, 2013, pp. 477--487.

The paper presents an empirical study that aims to corroborate the
relationship between the fault and change-proneness of APIs and the
degree of success of Android apps measured by their user ratings. For
this, the authors selected a sample of 7,097 free Android apps from the
Google Play Market and gathered information of the changes and faults
that the APIs used by them presented. Using this data and statistical
tools such as box-plots and the Mann-Whitney test, two main hypotheses
were analyzed. The first hypothesis tested the relationship between
fault-proneness (number of bugs fixed in the API) and the success of an
app. The second tested the relationship between change-proneness
(overall method changes, changes in method signatures and changes to the
set of exceptions thrown by methods) and the success of an app. Finally,
although no causal relationships between the variables can be assumed,
the paper found significant differences of the level of success of the
apps taking into consideration the change and fault-proneness of the
APIs they use.

\section{The Impact of API Change and Fault-Proneness on the User
Ratings of Android
Apps}\label{the-impact-of-api-change-and-fault-proneness-on-the-user-ratings-of-android-apps}

G. Bavota, M. Linares-Vásquez, C. E. Bernal-Cárdenas, M. D. Penta, R.
Oliveto and D. Poshyvanyk, in IEEE Transactions on Software Engineering,
vol.~41, no. 4, pp.~384-407, 1 April 2015. doi: 10.1109/TSE.2014.2367027

The paper by Bavota et al. aims to find empirical evidence supporting
the success of apps and the relationship with change- and
fault-proneness of the underlying APIs, where the success of the app is
measured by its user rating. They performed two case studies to find
quantitative evidence using 5848 free Android apps as well as an
explanation for these results doing a survey with 45 professional
Android developers. The quantitative case study was done by comparing
the user ratings to the number of bug fixes and changes in the API that
an app uses. They found that apps with a high user rating are
significantly less change- and fault-prone than APIs used by apps with a
low user rating. In the second case study the paper found that most of
the 45 developers observed a direct relationship between the user
ratings of apps and the APIs those apps use.

\section{How can i improve my app? Classifying user reviews for software
maintenance and
evolution}\label{how-can-i-improve-my-app-classifying-user-reviews-for-software-maintenance-and-evolution}

S. Panichella, A. D. Sorbo, E. Guzman, C. A. Visaggio, G. Canfora, and
H. C. Gall, in 2015 ieee international conference on software
maintenance and evolution (icsme), 2015, pp.~281--290.

The most popular apps in the app stores (such as Google Play or App
Store) receive thousands of user reviews per day and therefore it would
be very time demanding to go through the reviews manually to obtain
relevant information for the future development of the apps. This paper
uses a combination of Natural Language Processing Sentiment Analysis and
Text Analysis to extract relevant sentences from the reviews and to
classify them into the following categories: Information Seeking,
Information Giving, Feature Request, Problem Discovery, and Others. The
results show 75\% precision and 74\% recall when classifier (J48 using
data from NLP+SA+TA) is trained on 20\% of the data (1421 manually
labeled sentences from reviews of seven different apps) and the rest is
used for testing. The paper also states that the results do not differ
in a statistically significant manner when a different classifier is
used and shows that precision and recall can be further improved by
increasing the size of the data set.

\chapter{Release Engineering
Analytics}\label{release-engineering-analytics}

\section{Search Strategy}\label{search-strategy-1}

Release engineering is a relatively new research topic, given that
modern processes for releasing software (e.g.~continuous delivery) are
industry-driven. Therefore, we took an exploratory approach in
collecting any literature revolving around the topic of release
engineering from the perspective of software analytics. This will aid us
in determining a more narrow scope for our survey, subsequently allowing
us to find additional literature fitting this scope.

At the start of this project, five papers were given to us as a starting
point for the literature survey. These initial papers were
\citet{adams2016a}, \citet{da2016a}, \citet{d2014a}, \citet{khomh2012a},
and \citet{khomh2015a}.

We collected publications using two search engines: Scopus and Google
Scholar. These each encompass various databases such as ACM Digital
Library, Springer, IEEE Xplore and ScienceDirect. The queries we entered
are summarized in Figure 1. The publications found using this query
were:

\begin{itemize}
\tightlist
\item
  \citet{kaur2019a}
\item
  \citet{kerzazi2013a}
\item
  \citet{castelluccio2017a}
\item
  \citet{karvonen2017a}
\item
  \citet{claes2017a}
\item
  \citet{fujibayashi2017a}
\item
  \citet{souza2015a}
\item
  \citet{laukkanen2018a}
\end{itemize}

\begin{verbatim}
TITLE-ABS-KEY(
  (
    "continuous release" OR "rapid release" OR "frequent release"
    OR "quick release" OR "speedy release" OR "accelerated release"
    OR "agile release" OR "short release" OR "shorter release"
    OR "lightning release" OR "brisk release" OR "hasty release"
    OR "compressed release" OR "release length" OR "release size"
    OR "release cadence" OR "release frequency"
    OR "continuous delivery" OR "rapid delivery" OR "frequent delivery"
    OR "fast delivery" OR "quick delivery" OR "speedy delivery"
    OR "accelerated delivery" OR "agile delivery" OR "short delivery"
    OR "lightning delivery" OR "brisk delivery" OR "hasty delivery"
    OR "compressed delivery" OR "delivery length" OR "delivery size"
    OR "delivery cadence" OR "continuous deployment" OR "rapid deployment"
    OR "frequent deployment" OR "fast deployment" OR "quick deployment"
    OR "speedy deployment" OR "accelerated deployment" OR "agile deployment"
    OR "short deployment" OR "lightning deployment" OR "brisk deployment"
    OR "hasty deployment" OR "compressed deployment" OR "deployment length"
    OR "deployment size" OR "deployment cadence"
  ) AND (
    "release schedule" OR "release management" OR "release engineering"
    OR "release cycle" OR "release pipeline" OR "release process"
    OR "release model" OR "release strategy" OR "release strategies"
    OR "release infrastructure"
  )
  AND software
) AND (
    LIMIT-TO(SUBJAREA, "COMP") OR LIMIT-TO(SUBJAREA, "ENGI")
)
AND PUBYEAR AFT 2014
\end{verbatim}

\emph{Figure 1. Query used for retrieving release engineering
publications via Scopus.}

In addition to the search strategy that is based on a combination of
keywords and subject headings as described above, a review of the list
of publications that retrieved papers cite and are cited by is done.
These lists are provided by Google Scholar, as well as the reference
lists of the papers themselves. Results of which are listed in Table 1.

\emph{Table 1. Papers found indirectly by investigating citations of/by
other papers.}

\begin{longtable}[]{@{}lll@{}}
\toprule
Starting point & Type & Result\tabularnewline
\midrule
\endhead
\citet{souza2015a} & has cited & \citet{plewnia2014a}
\citet{mantyla2015a}\tabularnewline
\citet{khomh2015a} & is cited by & \citet{poo-caama2016a}
\citet{teixeira2017a}\tabularnewline
\citet{mantyla2015a} & is cited by & \citet{rodriguez2017a}
\citet{cesar2017a}\tabularnewline
\bottomrule
\end{longtable}

In order to aggregate our collective efforts, selected sources were
stored in a custom built web-based tool for conducting literature
reviews. The source code of this tool is published in a
\href{https://github.com/jessetilro/research}{GitHub repository}. By
commenting on and tagging our findings we were able to export a filtered
list of publications, the relevance of which was agreed upon.

\bibliography{book.bib}


\end{document}
